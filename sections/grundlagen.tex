\section{Grundlagen}

\rule{\linewidth}{0.5mm}
\subsection{Definition}
\begin{addmargin}[3cm]{0cm}

	Die Härte ist ein Mass für die Konzentration der im Wasser gelösten Ionen der Erdalkalimetalle. Im natürlichen Wasser sind vorwiegend Magnesiumionen und Calciumionen vorhanden.
	Weitere Erdalkalimetalle sind beispielsweise Strontium und Barium, von welchen auch Spuren im Wasser vorhanden sind.
	
	Man unterscheidet zwischen drei Härtebegriffen:
	
	\begin{table}[H]\centering
	\begin{tabular}{|p{4cm}|p{10cm}|}	
		\hline
		\bf{Gesamthärte}
			& $\mathrm{GH} = \mathrm{KH} + \mathrm{NKH}$ \\
		\hline
		\bf{Carbonathärte}
			& $\begin{aligned}
				\mathrm{KH} &= \frac{1}{2}\cdot c(\chemfig{HCO_{3}^{-}}) \\
							&= c(\chemfig{Ca}(\chemfig{HCO_3})_2) + c(\chemfig{Mg}(\chemfig{HCO_3})_2)
				\end{aligned}$ \\
		\hline
		\bf{Nichtcarbonathärte}
			& $\mathrm{NKH} = c(\chemfig{CaSO_{4}}) + c(\chemfig{MgSO_{4}}) + c(\chemfig{CaCl_{2}}) + c(\chemfig{MgCl_{2}}) +$...\\
		\hline
	\end{tabular}
	\caption[Zusammenhänge Härtetypen]{Zusammenhänge der Verschiedenen Härtetypen}
	\end{table}
	
	Es ist in der Schweiz immernoch üblich, in französischen Härtegraden zu rechnen. Die Einteilung geschieht wie folgt.
	
	\begin{table}[H]\centering
	\begin{tabular}{|m{4.1cm}|m{3.5cm}|m{6cm}|}	
		\hline
		\bf{Gesamthärte} $\Big(\frac{\mathrm{mmol}}{\mathrm{l}}\Big)$
			& \bf{Gesamthärte} $^\circ$fH
			& \bf{Bezeichnung Gesamthärte} \\
		\hline
		0 bis 0.7
			& 0 bis 7
			& sehr weich\\
		\hline
		$>$0.7 bis 1.5
			& $>$7 bis 15
			& weich \\
		\hline
		$>$1.5 bis 2.5
			& $>$15 bis 25
			& mittelhart \\
		\hline
		$>$2.5 bis 3.2
			& $>$25 bis 32
			& ziemlich hart \\
		\hline
		$>$3.2 bis 4.2
			& $>$31 bis 42
			& hart \\
		\hline
		$>$4.2  
			& $>$42
			& sehr hart \\
		\hline
	\end{tabular}
	\caption[Härtegrade]{Zusammenhänge zwischen Härtegraden und deren Bezeichnungen}
	
	\end{table}
\end{addmargin}


\rule{\linewidth}{0.5mm}
\subsection{Titration}
\begin{addmargin}[3cm]{0cm}
	Titration ist eine oft verwendete Methode, um die unbekannte Konzentration eines Stoffes in einer Lösung zu ermitteln. In diesem Vorgehen wird eine Masslösung (auch Titrator oder Titrant genannt) mit bekannter Konzentration in die Probelösung (Titrand) gegeben, bis die Äquivalenzstoffmenge erreicht ist. Dieser Punkt erkennt man je nach Art der Lösung unterschiedlich. Bei unserem Verfahren, der Säure-Base-Titration, erfolgt aufgrund der Versetzung mit einem Farbkomplex ein plötzlicher Farbumschlag.
	
	Arten der Endpunktserkennung sind:
	\begin{itemize}
		\item Niederschläge \\ Beispiel: Fällungstitration, nutzt die Reaktion von \chemfig{Ag^{+}}-Ionen und \chemfig{Cl^{-}}-Ionen aus, bei welcher sich zum Endpunkt milchiger Niederschlag bildet.
		
		\item Änderung des elektrischen Potentials \\ Beispiel: Potentiometrische Titration, erfolgt mit zwei Elektroden
		
		\item Änderung des pH-Werts \\ Beispiel: Säure-Base Titration
		
		\item Änderung des Stromflusses bei konstanter angelegter Spannung \\ Beispiel: Biamperometrie
		
		\item Änderung der Temperatur \\ Beispiel: Thermometrische Titration
	\end{itemize}
\end{addmargin}
 
 
\rule{\linewidth}{0.5mm}
\subsection{Praxisbezug}
\begin{addmargin}[3cm]{0cm}
	Auch wenn Trinkwasser für die Versorgung des Körpers mit Magnesium und Calcium eine untergeordnete Rolle spielt, sind diese Elemente für den menschlichen Organismus essentiell.
	
	Der Härtegrad des Wassers macht sich zum grössten Teil im Leitungswasser bemerkbar. Ein hoher Härtegrad führt zur Verkalkung der Leitungen und Haushaltsgeräten etc.
	
	Weiches Wasser ist vor allem dort vorteilhafter, wo das Wasser erhitzt wird oder auch für die Pflanzenbewässerung verwendet wird. 
\end{addmargin}